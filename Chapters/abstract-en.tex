%!TEX root = ../template.tex
%%%%%%%%%%%%%%%%%%%%%%%%%%%%%%%%%%%%%%%%%%%%%%%%%%%%%%%%%%%%%%%%%%%%
%% abstrac-en.tex
%% NOVA thesis document file
%%
%% Abstract in English([^%]*)
%%%%%%%%%%%%%%%%%%%%%%%%%%%%%%%%%%%%%%%%%%%%%%%%%%%%%%%%%%%%%%%%%%%%

\typeout{NT FILE abstrac-en.tex}%

The number of services and applications that require and rely on transactional, replicated and verifiable data to 
function is increasing with each passing day, from banking and financial applications to online voting. With these
requirements also come challenges, like availability, consistency, and security mechanisms that allow for integrity, non-repudiation and
encryption of messages.

A common solution that these applications use to satisfy these requirements are blockchain protocols, usually defined as
distributed ledgers with a growing list of records (blocks), linked together by cryptographic hashes. Records are permanent and
usually distributed across a peer-to-peer computer network where participants adhere to the consensus protocols to validate and add
new transactions.

There are a wide range of different blockchain protocols and some of them are not set in stone, Like Ethereum which moved from proof of work
into a proof of stake solution or Tezos where stakeholders are capable of proposing and agreeing on changes to the consensus protocol allowing
it to evolve over time. Because of this a number of tools focused on assisting with the development of these protocols have emerged.

One of these tools is MOBS, Modular Blockchain Simulator, a simulator built with extensibility and
modularity in mind, allowing users to simulate any family of protocols as well as parametrize multiple scenarios for their study.
These parameterizations include bandwidth limits, byzantine behaviour of the participant nodes and adversarial behaviour. After the execution
the desired statistics and information needed to validate the execution of the protocols can also be parametrized.

The goal of this dissertation is of tools that will allow us to validate the correctness 
of the implemented consensus protocols through the logs of the simulator, allowing for better qualitative results to be extracted and evaluated.
Secondly we also propose an extension to this tool to better allow the simulator to provide different sets of execution environments by
allowing parameterization of the network layer of the simulator we propose Chord, HyParView and X-BOT since they are diverse protocols and very different from each other.


% Palavras-chave do resumo em Inglês
% \begin{keywords}
% Keyword 1, Keyword 2, Keyword 3, Keyword 4, Keyword 5, Keyword 6, Keyword 7, Keyword 8, Keyword 9
% \end{keywords}
\keywords{
  Blockchain,
  Networking,
  Consensus,
  Simulation,
  Validation of protocol properties
}
