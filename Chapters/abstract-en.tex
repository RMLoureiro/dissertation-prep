%!TEX root = ../template.tex
%%%%%%%%%%%%%%%%%%%%%%%%%%%%%%%%%%%%%%%%%%%%%%%%%%%%%%%%%%%%%%%%%%%%
%% abstrac-en.tex
%% NOVA thesis document file
%%
%% Abstract in English([^%]*)
%%%%%%%%%%%%%%%%%%%%%%%%%%%%%%%%%%%%%%%%%%%%%%%%%%%%%%%%%%%%%%%%%%%%

\typeout{NT FILE abstrac-en.tex}%

The number of services and applications that require and rely on transactional, replicated and verifiable data to 
function is increasing with each passing day, from banking and financial applications to online voting. With these
requirements also come challenges, like availability, consistency, and security mechanisms that allow for integrity, non-repudiation and
encryption of messages.

A common solution that these applications use to satisfy these requirements are blockchain protocols, usually defined as
distributed ledgers with a growing list of records (blocks), linked together by cryptographic hashes. Records are permanent and
distributed across a peer-to-peer computer network where participants adhere to the consensus protocols to validate and add
transactions.

There are a wide range of different blockchain protocols and some of them are not set in stone, Like Ethereum which moved from proof of work
into a proof of stake solution or Tezos where stakeholders are capable of proposing and agreeing on changes to the consensus protocol allowing
it to evolve over time. Because of this a number of tools focused on assisting with the development of these protocols have emerged.

One of these tools is MOBS, Modular Blockchain Simulator, built with extensibility and
modularity in mind, allows for simulation of consensus and blockchain protocols as well as parametrize multiple scenarios for their study which
include bandwidth limits, network layout and Byzantine and adversarial behaviour of participant nodes.

The goal of this dissertation is twofold, first we want to extract statistics and information from MOBS' logs so that we can provide
developers with qualitative information about the protocols' execution to empirically verify the protocols' properties
(it does not avoid the need for formal verification, it is only a support during the prototyping phase). 
Secondly we also propose to improve the
behaviour and parameterization of the network layer to allow for specific membership protocols to be selected and determine
the network layout and re-configuration when participants join or leave the network at runtime.

% Palavras-chave do resumo em Inglês
% \begin{keywords}
% Keyword 1, Keyword 2, Keyword 3, Keyword 4, Keyword 5, Keyword 6, Keyword 7, Keyword 8, Keyword 9
% \end{keywords}
\keywords{
  Blockchain,
  Networking,
  Consensus,
  Simulation,
  Validation of protocol properties
}
