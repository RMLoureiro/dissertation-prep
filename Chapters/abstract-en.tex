%!TEX root = ../template.tex
%%%%%%%%%%%%%%%%%%%%%%%%%%%%%%%%%%%%%%%%%%%%%%%%%%%%%%%%%%%%%%%%%%%%
%% abstrac-en.tex
%% NOVA thesis document file
%%
%% Abstract in English([^%]*)
%%%%%%%%%%%%%%%%%%%%%%%%%%%%%%%%%%%%%%%%%%%%%%%%%%%%%%%%%%%%%%%%%%%%

\typeout{NT FILE abstrac-en.tex}%

The number of services and applications that require and rely on transactional, replicated and verifiable data to 
function is increasing with each passing day, from banking and financial applications to online voting. With these
requirements also come challenges, like availability, consistency, and security mechanisms that allow for integrity, non-repudiation and
encryption of messages.

A common solution that these applications use to satisfy these requirements is Distributed Ledger Technologies 
or DLT. These systems are characterized as having a decentralized database, consensus mechanisms 
to validate transactions and immutable data once verified. But with the wide range of different requirements come a wide range protocols.
These require testing, validation, and inevitably come vulnerabilities or errors that need to be corrected.
Since the use of DLTs is recently new there is a lack of tools for the testing and
validation of these protocols, which means that problems with the logic of the algorithm or vulnerabilities are often discovered in live
applications.

One of these tools is MOBS \url{https://github.com/mce-alves/MOBS}, Modular Blockchain Simulator, a simulator built with the extensibility and
modularity in mind, allowing users to simulate any family of protocols as well as parametrize multiple scenarios for their study.
These parameterizations include bandwidth limits, byzantine behavior of the participant nodes and adversarial behavior. After the execution
the desired statistics and information needed to validate the execution of the protocols can also be parametrized.

The goal of this dissertation is twofold we propose an extension to this tool to better allow the simulator to provide different sets of execution environments by
allowing parameterization of the network layer of the simulator we propose Chord, HyParView and X-BOT since they are diverse protocols and very different from each other.
Secondly also propose the implementation of tools that will allow us to validate the correctness of the implemented consensus protocols through the logs of the simulator,
allowing for better qualitative results to be extracted and evaluated.

% Palavras-chave do resumo em Inglês
% \begin{keywords}
% Keyword 1, Keyword 2, Keyword 3, Keyword 4, Keyword 5, Keyword 6, Keyword 7, Keyword 8, Keyword 9
% \end{keywords}
\keywords{
  Distributed Ledger Technology,
  Blockchain,
  Networking,
  Consensus,
  Simulation,
  Validation of protocol properties
}
