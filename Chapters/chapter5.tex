%!TEX root = ../template.tex
%%%%%%%%%%%%%%%%%%%%%%%%%%%%%%%%%%%%%%%%%%%%%%%%%%%%%%%%%%%%%%%%%%%%
%% chapter5.tex
%% NOVA thesis document file
%%
%% Chapter with a short latex tutorial and examples
%%%%%%%%%%%%%%%%%%%%%%%%%%%%%%%%%%%%%%%%%%%%%%%%%%%%%%%%%%%%%%%%%%%%

\typeout{NT FILE chapter5.tex}%

\chapter{Work to be developed}\label{cha:work_to_be_developed}

This chapter will describe the work that will be developed to address the
problems presented in~\ref{sub:problem} making use of the tools and information
presented in~\ref{cha:related_work} and~\ref{cha:mobs} and will also provide
a calendarization of the proposed work.

\section{Proposed work}\label{sub:sub_proposed_work}

The work will be divided in two sections, extending the logs and statistics 
module to allow for better qualitative
information about the implemented protocols at the node level and enhancing MOBS' network layer.

\subsection{Analyse data qualitatively}\label{subsec:improve_qualitative_data}

In a first step to better understand what information we can extract and how
we can leverage the simulator to evaluate Blockchain protocols, we will finish implementing
simpler consensus protocols, namely Chandra-Toueg and MultiPaxos described in~\ref{sec:consensus}.
Finishing the implementations will give us the ability to test with more than 1 proposer and
failing and malicious nodes.

Extracting more qualitative data will allow developers to better analyse the properties
from the implemented protocols and compare their executions in different scenarios and conditions,
this does not avoid the need for formal verification, but it can be a powerful tool in prototyping stage.
In a first phase we will provide tools to analyse the output of consensus protocols
with metrics that can be transversal across different classical consensus protocols, like
average time needed for consensus and percentage of accepting nodes. These metrics were considered
for the scenarios implemented until now that are optimistic and do not account for Byzantine or
faulty nodes, more work needs to be done to consider what more metrics need to be extracted under
imperfect conditions.

This work should allow us to better understand and see if changes
need to be done to the current log and statistics modules in order to obtain enough qualitative
data for the consensus protocols to evaluate and validate their execution under different
conditions and scenarios.

We will leverage the post execution script approach to scrub and analyse the logs from MOBS,
this will allow the users to use custom scripts to fit their needs. We will also consider an
improvement to the GUI to allow the user to select the script they want to execute and in turn
feed data into the GUI for visual representation.

After these implementations and validations we will move to blockchain protocols and see
if we can qualitatively evaluate these protocols, specifically we will try to
replicate the vulnerabilities found in Ethereum in~\cite{ethereum_analysis}, and try
to implement Solana and see if we can reach similar conclusions to what was found in~\cite{solana_halting_problem}.

\subsection{Network Layer Enhancement}\label{subsec:network_layer_enchancement}

In this section we will present the proposed enhancements to the Network layer of
MOBS. For this, we propose the following work to be conducted:

\begin{itemize}
  \item \textbf{Modular Network Layer:} The goal is to make the network layer
modular and allow for the implementation of diverse membership protocols. This
will allow for a more diverse set of parametrizations and network environments
when testing the already implemented.
  \item \textbf{Network Module Parametrization:} Here we want to expand the already
provided parametrizations of this module, allowing for the parametrization of events
such as, allowing nodes to join the network after the simulation already started,
allowing a random amount of nodes to fail at a random point during execution and
simulate periodic bandwidth slowdowns between certain nodes.
\end{itemize}

These enhancements to the Network module will allow not only for the advantages
described above, but also allow for the network itself to restructure in case of
node failures and a closer replication of real-life conditions, since right now the
overlay layout is static, besides being able to parametrize nodes to be offline after
a certain amount of time. In the real world if this happens the network will restructure
and initiate protocols to keep its integrity and correctness, which depends on the
protocol that was implemented, for Chord we would try to re-structure back to a cyclic
ring formation, for HyParView we would populate each node's active vies to go back to the
parametrized size.

For this we propose the implementation of three membership overlay protocols, Chord,
HyParView and X-BOT. These three protocols were chosen so that we can have a
structured overlay protocol, an unstructured protocol and a partially-structured
respectively.

We also propose a small upgrade to the GUI, allowing users to select the network
protocol that they desire their simulation to run on alongside with the option for
a validation script to be run after to check the network's correctness and integrity.
These two extensions will allow users to either check the validation of a network
protocol if they choose to implement a new one, or make sure of the integrity of
the network overlay when studying the logs from the protocols that run on top of it.

We propose to evaluate this work in the following way:

\begin{itemize}
  \item After implementation of these protocols a script in Python with the aid of
the NetworkX package should be written in order to validate the overlay's properties.
The script should be able to generate a graph from the logs provided by MOBS.
The values of their properties and the metrics we can extract the overlay should be
comparable to the ones presented in protocol's paper and other tests that were conducted
on them.
  \item After validation of each of these protocols run the already implemented
consensus and blockchain protocols, and we should see similar results to the ones obtained
when testing with the current Network layer. We should expect some variance in the 
performance of the protocols implemented at the Node module since one of the
goals of this enhancement is to study their performance using different overlays.
  \item The enhancements to the GUI should be seamless and work for existing scenarios
while now giving the ability to select which network protocol to run, the validation
script for said protocol and show the results given by that script.
\end{itemize}


\section{Work Plan}\label{sub:work_plan}

In this section we present the work plan for the elaboration of this work.
We will divide the work into six phases. The enhancement of the network layer,
study of the values and parameters needed for the validation of consensus protocols,
implementation of consensus protocols and their validation, implementation of blockchain
protocols and their validation, writing an article detailing the work conducted and
finally writing of the final thesis document.

\begin{table}[ht]
\centering
\caption{Schedule}\label{tab:schedule}
\begin{tabular}{lccc}
  \toprule
  \multicolumn{1}{c}{\textbf{Task}}					& \textbf{Start Date} & \textbf{End Date} & \textbf{Weeks}	\\ 
  \midrule
  \textbf{Consensus protocols}						& 11 March             & 3 May       & 8              	\\
  \quad Study of consensus protocols in MOBS 								&                     &                   &                	\\ 
  \quad Implementation 								&                     &                   &                	\\ 
  \quad Testing and Validation								&                     &                   &                	\\ 
  \midrule
  \textbf{Enhancement of the network layer}					& 6 May            & 14 June          & 6             	\\
  \quad Implementation 								&                     &                   &                	\\ 
  \quad Testing and Validation 								&                     &                   &                	\\
  \midrule
  \textbf{Blockchain protocols}						& 17 June             & 16 August       & 10         	\\
  \quad Study of needed parameters for evaluation 								&                     &                   &                	\\ 
  \quad Implementation 								&                     &                   &                	\\ 
  \quad Testing and Validation								&                     &                   &                	\\ 
  \midrule
  \textbf{Thesis plus Article}									& 19 August           & 30 September      & 6          	  	\\   
  \bottomrule
\end{tabular}
\end{table}

On our planning we are leaving about 2 months as a buffer in case any of the
phases of work becomes more troublesome than expected.