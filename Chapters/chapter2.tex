%!TEX root = ../template.tex
%%%%%%%%%%%%%%%%%%%%%%%%%%%%%%%%%%%%%%%%%%%%%%%%%%%%%%%%%%%%%%%%%%%%
%% chapter2.tex
%% NOVA thesis document file
%%
%% Chapter with the template manual
%%%%%%%%%%%%%%%%%%%%%%%%%%%%%%%%%%%%%%%%%%%%%%%%%%%%%%%%%%%%%%%%%%%%

\typeout{NT FILE chapter2.tex}%

\chapter{Related work}
\label{cha:related_work}

In this chapter we will discuss relevant work considering the goals of the work
to be conducted in this thesis. In particular, we will focus on the following topics:

In section \ref{sec:membership_protocols} we discuss the differences between
the different kinds of membership systems as well as some implementations.

In section \ref{sec:babel} we will discuss Babel(insert reference), a framework
to develop distributed protocols.

In section \ref{sec:protocol_analysis_tools} we will present and compare some 
protocol analysis tools and compare them to MOBs.

\section{Membership Protocols}
\label{sec:membership_protocols}

This section introduces some membership protocols that have been considered to
part of the work that will be developed.


\subsection{Membership}
\label{sub:membership}

A protocol where each node knows every other node in the network might work well
for small networks, but it is not a scalable solution since each node would need
to have \textit{n - 1} communications channels open at all time, being n the
amount of nodes in the system, and each node needs to be following any and all
changes to the system. This is not feasible since the number of links between
nodes would rise quadratically and networks can easily reach thousands of
participants. In order to overcome the challenges that arise from large networks,
we use partial view membership protocols. In these protocols each node only knows
and maintains information about a small selection of nodes in the systems,
making it a more scalable strategy than a total view protocol, since
the number of connections grow at a logarithmic rate instead.

When maintaining a partial view, protocols usually follow one of two strategies
when managing their memberships:

\begin{itemize}
  \item \textbf{Reactive:} With this strategy the partial view only changes when
the membership suffers changes, usually by a node leaving or joining the membership
  \item \textbf{Cyclic:} With this approach, the partial view changes periodically.
Every \textit{t} seconds nodes exchange information that may lead to changes to the
partial view.
\end{itemize}

This membership protocols may be represented by a graph where the vertices represent
the participants of the networks and the edges represent the communication links.
Depending on how the graph ends up we can classify the membership as structured,
unstructured or partially structured. In the next sections we will go over these
types of memberships and some implementations.

\subsection{Structured Overlays}
\label{sub:structured_overlays}

Structured overlay memberships follow a pre-defined structure by having the nodes
routed to a specific logical position in the network. This known structure allows
improvement on search primitives, enabling efficient discovery of data and process.
This position is usually determined by giving a unique identifier for each node of the
network.

However, having a predefined topology come at the cost of more costly re-structuring
usually by a node leaving the network or a new participant joining. This process
is usually slow and costly since a change of one node in the membership may affect
the network at a global level. This drawback becomes more relevant in high churn rate
scenarios.

\subsubsection{Chord}
\label{subsec:chord}

Consistent Hashing and Random Trees, or Chord(referencia) is a Distributed Hash
Table based algorithm where each node is assigned a unique identifier based on
the output of a hash function. Usually this has is based on the IP address of
the node since this is unique for every node of the network, and making it so
that once a node joins the network their identifier is set in stone. These
identifiers are also so that each node knows which keys of the distributed hash
table are assigned to them, since the range of keys that belong to a certain
node is represented by the range defined by their identifier up to the
identifier of the next node.

The graph generated based on the membership overlay will be a cyclic graph with
a ring shape, with the nodes ordered by their generated identifier. Each node
also maintains a routing table with around O(log \textit{n}) distinct entries
called a finger table. This routing table is used in order to navigate the overlay
in a more efficient manner.

The tables maintained by these nodes are automatically updated when a new node joins
or leaves the system  making it always possible to find a node responsible to a given
key. However, simultaneous failures may break the overlay since the correctness
of the membership depends on each node knowing their correct neighbors. In order
to increase fault tolerance, a periodical stabilization protocol is run in background,
making sure the neighbors are still active and restructuring the overlay
accordingly as well as updating the entries on the finger table.

\subsubsection{Pastry}
\label{subsec:pastry}

Pastry(insert reference), similarly to Chord is a structured overlay protocol that
defines a distributed hash table. A Pastry system is a self-organizing overlay
network of nodes where every node has a 128-bit identifier. This identifier
is assigned randomly when a node joins the system and is used to indicate a
node's position in a circular space that ranges from 0 to $2^{128}-1$. It is assumed
that the hash function that gives the node's identifier generated a uniform
distribution of identifiers around the 128-bit space.

Each Pastry node maintains a \textit{routing table}, a \textit{neighborhood set}
and a \textit{leafing set}. The routing table contains the entries based
on substrings of the own node's identifier. The neighborhood set contains
the node identifiers and the corresponding IP address of the \textit{M} the closest
nodes to the local node, \textit{M} being parameterizable.  The neighborhood
set contains the nodes that are used to maintain local properties. The leafing
set contains the \textit{L}/2 closest large node identifiers and \textit{L}/2
smallest node identifiers, \textit{L} also being a parametrized value. The
leaf set is used in message routing.

Pastry makes use these tables in the following fashion: when a message is received
for another node in the network we check the leaf set, if the node is in range of
the leaf set we route directly to the destination node, otherwise we check
the routing table for a node that shares a common prefix with the key.

\subsection{Unstructured Overlays}
\label{sub:unstructured_overlays}

Unstructured overlays place few constrains as on how the nodes chose their
neighbors. This results in a random graph that is hard to predict and describe.

By not having a structured overlay these memberships end up being more fault-tolerant
in high churn rate scenarios due to the cost of the network restructuring
itself is fairly low.

\subsubsection{SWIM}
\label{subsec:swim}

SWIM(referencia) stands for \textbf{S}calable \textbf{W}eakly-consistent \textbf{I}nfection-style
Process Group \textbf{M}embership Protocol. This protocol is composed of two distinct
components, a failure detector and a dissemination component.

Unlike traditional gossip based overlays that rely on a heartbeat strategy,
the failure detector in SWIM is independent of the rest of the protocol.
The failure detector is fully decentralized and is executed in a randomized
probe-based fashion, the authors later suggest an optimization via a round-robin
fashion instead. Every \textit{t} seconds, parameterizable value, a random
node that belongs to the membership is checked to see if its still alive, if
no awnser is received the request is forwarded to \textit{k} other members to
execute an indirect probe. If the nodes responds to at least one of the indirect
requests the suspicion is removed and an \textit{alive} message is spread to
fasten the process in case other nodes suspect from it.  This mechanism reduces The
number of false positives in failure detection, which can usually be due to a
temporary unavailability from a network partition or from a slow-down of a node
due to a high load of requests. As a drawback failure detection is slower.

The other component of this protocol is the gossip dissemination system which
maintains a partial view of the network. This component updates whenever a member
joins or leaves the system by an infection style dissemination protocol. In order to
make this more efficient, the updates piggyback the messages that are sent during the
failure detection procedure.

\subsubsection{HyParView}
\label{subsec:hyparview}

HyParView (referencia) is a gossip based membership protocol that offers high
resilience and high delivery reliability of messages while being highly scalable.
It relies on a hybrid approach by maintaining, besides the active view that is 
used by the nodes to maintain the communication overlay, a passive view, in case
the networks needs to be restructured.

HyParView uses different approaches when it comes to maintaining each of the views,
for active view a reactive strategy is used, nodes react to events that require 
the network to be restructured such as new nodes joining the membership or
existing ones leaving, either by failing or by choice.

The nodes in the active view are the ones with which each node maintains a
communication link. In case the active view needs to be changed the nodes in
the passive view may be promoted to active nodes the same way nodes in the
active view may be demoted to the passive view. All the nodes in the active
view of a given node have that node in their active view, making the connection
graph that represents the overlay be a bidirectional graph.

For the passive view, HyParView uses a cyclic strategy by periodically exchanging
information with other nodes regarding the contents of its views. This information
is exchanged by means of a shuffle operation with another node from its active view.
After the exchange, the nodes integrate the new received members in its
passive views, even if for that, some older members have to be dropped. This
mechanism helps maintaining a homogeneous overlay in the number of connections
each node has.

The gossip strategy relies on the use of TCP as a reliable transport protocol and fail
detector. Since all members are tested at each step of the gossip protocol, failed nodes
do not stay unnoticed for long. This way, HyParView provides with fast self-healing
properties that are characteristic of gossip protocols.

Tests done in (referencia hyparview) show that the algorithm is able to recover from as much as 80\%
node failure, as long as the overlay stays connected. By being able quickly react to failures
in the system, the protocol was shown to be able to maintain 100\% reliability 
for message dissemination.


\subsection{Partially Structured Overlays}
\label{sub:partially_structured_overlays}

Partially structured overlays aim to get the best of both strategies. We can 
leverage the easy to maintain and fault-tolerant unstructured overlays and
by applying some optimization procedure to the network we can achieve a more
efficient search and application level routing.

\subsubsection{T-MAN}
\label{subsec:t-man}

T-MAN (referencia) was created with the motivation to give
the ability of taking some random overlay, and \textit{evolve} it into another one.

The logic behind the algorithm is giving each node a ranking value that every node can
use to apply a function to determine how desirable a node is as a neighbor.
Each node maintains a partial view that contains the addresses of nodes that are not its
immediate neighbors, much like the HyParView overlay described in \ref{subsec:hyparview}.
Periodically each node exchanges its partial view with the first node in its active view,
according to the ranking
values which depends on the target overlay. The receiver will execute the same procedure
as the sender so they can later merge their local views and apply the ranking function.
Using their peers' views to improve their own the overlay will gradually become closer
the desired overlay.

Experimentally this algorithm is shown to be scalable and fast, with the convergence
times growing approximately at a logarithmic rate in function of the number of nodes in
the network. The problem that surges with this, is that the network only becomes as 
fault-tolerant as the desired overlay, since T-MAN doesn't aim to maintain a balanced degree
between the nodes, this might create uneven load balance or even node isolation during
or after the procedure

\subsubsection{X-BOT}
\label{subsec:x-bot}

X-BOT (referencia) stands for \textbf{B}ias the \textbf{O}verlay \textbf{T}opology
according to some targeting criteria \textbf{X}. This protocol is completely decentralized,
and the nodes do not require any prior knowledge of where they will end up in the final topology. 
This protocol strives to preserve the degree of the nodes that participate in the 4-node coordinated
optimization technique, described in greater detail below, this is essential to preserve
the connectivity of the overlay. X-BOT is built in a way that every modification that is
done by the protocol increases its efficiency and due to the dynamic nature of the model,
its ensured that the overlay does not stabilize in a local minimum. These optimizations
are done in a way that key features of the overlay, such as low clustering coefficient and
low overlay diameter, are preserved. The protocol is highly flexible because it relies on a
companion oracle to estimate the link cost and therefore bias the network according to
different cost metrics.

The companion oracle is accessible by all nodes, and its sole purpose is to give the
link cost from the node that invokes it to a given node.

The 4-node coordinated technique that has been referred above works as follow, a
node \textit{i}, the initiator, starts the optimization round selecting a node from
its passive view. Node \textit{O} is a node from \textit{i}'s active view that
will be replaced. Node \textit{c}, the candidate, is a node from \textit{i}'s passive
view that is going to be upgraded to its active view. And finally node \textit{d}
is the node to be removed from the candidate's view so that \textit{i} can be accepted.
These nodes are always selected based upon the link cost values provided by the oracle,
ensuring that every time the optimization procedure is called the network increases its
efficiency

\section{Babel}
\label{sec:babel}

In this section we will discuss Babel (referência), a framework that simplifies the
development of distributed protocols within a process that executes in real hardware.
Babel simplifies the development process by abstracting the developer from the
low level complexities associated with typical distributed systems implementations like
interactions with other protocols, communication details, handling timers and concurrency control.
The communication comlpexities are hidden behind abstractions called \textit{channels}
that can be extended by the developer. The Babel framework is composed by three main components:

\begin{itemize}
  \item \textbf{Protocols:} This is the component the developer implements encoding
the behavior of the distributed system being designed. Protocols are modeled as
state machines whose state changes by action of external events, for this purpose
each protocol contains an event queue that keeps track of these events.
  \item \textbf{Core:} This is the central component that coordinates the
execution of all protocols and mediates all interactions within the framework.
Core also keeps track of timers setup by the protocols and delivers an event to the
correct protocol when the timer is up.
  \item \textbf{Network:} This component handle the channel abstraction that was
mentioned previously. Protocols can interact with channels by opening and closing connections
and sending messages as well as receiving relevant events for the protocol.
\end{itemize}

Babel is provided as a java library and the developer can implement a protocol
by extending the abstract class \textit{GenericProtocol} that provides the needed
methods to generate events, register callbacks to process received events and 
the initial behavior for a node for the protocol that is being implemented.
The provided API can be split into three categories:

\begin{itemize}
  \item \textbf{Timers:} This allows for the execution of periodic actions.
  \item \textbf{Inter-protocol communication:} Babel supports multiple protocols
executing concurrently in the same process so mechanisms for these protocols
to interact with each other exist. These mechanisms take the form of requests and
notifications.
  \item \textbf{Networking:} Babel offers different network channels with different
capabilities but keeping the same model of interactions between protocols and the
different channel implementations.
\end{itemize}

Evaluation testing in (referencia) show that Babel can reduce the effort for programmers
in creating prototypes of distributed protocols, comparing the number of lines programmed
using Babel and standalone versions of the same protocols in java. This is achieved
mostly by the logic that is covered in the network and core layers, relieving the developer
from having to worry about handling timers, message ordering and most communication besides events.

Regarding performance tests conducted by implementing MultiPaxos (referencia no artigo do babel) with the framework
when comparing to standalone implementations, the implementation using Babel showed comparable performance
with less implementation overhead.

\section{Protocol Analysis Tools}
\label{sec:protocol_analysis_tools}
