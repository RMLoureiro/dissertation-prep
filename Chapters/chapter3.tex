%!TEX root = ../template.tex
%%%%%%%%%%%%%%%%%%%%%%%%%%%%%%%%%%%%%%%%%%%%%%%%%%%%%%%%%%%%%%%%%%%%
%% chapter3.tex
%% NOVA thesis document file
%%
%% Chapter with the template manual
%%%%%%%%%%%%%%%%%%%%%%%%%%%%%%%%%%%%%%%%%%%%%%%%%%%%%%%%%%%%%%%%%%%%

\typeout{NT FILE chapter3.tex}%

\chapter{Related work}
\label{cha:related_work}

\section{Babel}
\label{sec:babel}

In this section we will discuss Babel \cite{babel}, a framework that simplifies the
development of distributed protocols within a process that executes in real hardware.
Babel simplifies the development process by abstracting the developer from the
low level complexities associated with typical distributed systems implementations like
interactions with other protocols, communication details, handling timers and concurrency control.
The communication comlpexities are hidden behind abstractions called \textit{channels}
that can be extended by the developer. The Babel framework is composed by three main components:

\begin{itemize}
  \item \textbf{Protocols:} This is the component the developer implements encoding
the behavior of the distributed system being designed. Protocols are modeled as
state machines whose state changes by action of external events, for this purpose
each protocol contains an event queue that keeps track of these events.
  \item \textbf{Core:} This is the central component that coordinates the
execution of all protocols and mediates all interactions within the framework.
Core also keeps track of timers setup by the protocols and delivers an event to the
correct protocol when the timer is up.
  \item \textbf{Network:} This component handle the channel abstraction that was
mentioned previously. Protocols can interact with channels by opening and closing connections
and sending messages as well as receiving relevant events for the protocol.
\end{itemize}

Babel is provided as a java library and the developer can implement a protocol
by extending the abstract class \textit{GenericProtocol} that provides the needed
methods to generate events, register callbacks to process received events and 
the initial behavior for a node for the protocol that is being implemented.
The provided API can be split into three categories:

\begin{itemize}
  \item \textbf{Timers:} This allows for the execution of periodic actions.
  \item \textbf{Inter-protocol communication:} Babel supports multiple protocols
executing concurrently in the same process so mechanisms for these protocols
to interact with each other exist. These mechanisms take the form of requests and
notifications.
  \item \textbf{Networking:} Babel offers different network channels with different
capabilities but keeping the same model of interactions between protocols and the
different channel implementations.
\end{itemize}

Evaluation testing in \cite{babel} show that Babel can reduce the effort for programmers
in creating prototypes of distributed protocols, comparing the number of lines programmed
using Babel and standalone versions of the same protocols in java. This is achieved
mostly by the logic that is covered in the network and core layers, relieving the developer
from having to worry about handling timers, message ordering and most communication besides events.

Regarding performance tests conducted by implementing MultiPaxos \cite{paxos} with the framework
when comparing to standalone implementations, the implementation using Babel showed comparable performance
with less implementation overhead.

\section{NetworkX}
\label{sec:networkx}

NetworkX \cite{networkx} is a network analysis Python package that allows for the analysis of
networks and network algorithms. This package provides data structures for representing
several types of graphs such as simple graphs, directed graphs and graphs with parallel
edges and self-loops. In addition to the basic data structures, there are also
graph algorithms implemented for calculation network properties and structure measures
such as node connectivity, clustering coefficient, average shortest path.

For NetworkX nodes can be any time of hashable object such as strings, numbers,
files or even functions. Edges or links are represented as a tuple of nodes. The
standard ways of representing graphs are edge lists, adjacency matrices and adjacency
lists. Functions to manipulate graphs are also provided, simple functions come as class
methods, like adding or removing nodes or edges and complex statistics and algorithms
are provided as package functions such as clustering, shortest path and visualization.

Another of the implemented algorithms NetworkX provides is the ability to draw
a given graph, will be a helpful visual aid to understand the layout of the network
that a protocol generates.

\section{Simulators}
\label{sec:simulators}

\subsection{VIBES}
\label{subsec:vibes}

\subsection{BlockSim}
\label{subsec:blocksim1}

\subsection{BlockSim}
\label{subsec:blocksim2}

\subsection{SimBlock}
\label{subsec:simblock}

\subsection{JABS}
\label{subsec:jabs}