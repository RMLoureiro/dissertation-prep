%!TEX root = ../template.tex
%%%%%%%%%%%%%%%%%%%%%%%%%%%%%%%%%%%%%%%%%%%%%%%%%%%%%%%%%%%%%%%%%%%%
%% abstrac-pt.tex
%% NOVA thesis document file
%%
%% Abstract in Portuguese
%%%%%%%%%%%%%%%%%%%%%%%%%%%%%%%%%%%%%%%%%%%%%%%%%%%%%%%%%%%%%%%%%%%%

\typeout{NT FILE abstrac-pt.tex}%

O número de serviços e aplicações que necessitam e utilizam dados replicados, verificáveis e transacionais para funcionarem estão a aumentar
com cada dia que passa, desde aplicações bancárias e financeiros a sistemas de voto online. Com estes requisitos também vêm desafios,
como disponibilidade, consistência dos dados e mecanismos de segurança que permitam a integridade, a não repudiação e a encriptação de mensagens.

Uma solução comum que estas aplicações usam para satisfazer estes requisitos são Tecnologias de Registos Distribuídos ou TRD. Os sistemas de TRD
são caracterizados por terem uma base de dados descentralizada, mecanismos de consensos para  validar transações e dados imutáveis após verificados.
Mas com a vasta diversidade de requisitos vem uma quantidade diversa de protocolos. Estes necessitam de ser testados, validados e inevitavelmente 
corrigir os erros de lógica e as vulnerabilidades descobertas 'a posteriori'. Como o uso de TRDs é relativamente recente, há uma falta de ferramentas para
testar e validar estes protocolos, o que implica que problemas de lógica ou vulnerabilidades são muitas vezes descobertos depois das aplicações que os utilizam
serem lançadas.

Uma destas ferramentas é o MOBS(referencia), Modular Blockchain Simulator, este simulador foi construído com a extensibilidade e modularidade em mente,
permitindo os utilizadores simular qualquer família de protocolos tal como parametrizar múltiplos cenários para o estudo destes. Estas parametrizações incluem
limites de bandwidth, comportamentos bizantinos dos nós participantes e comportamento adversarial. Após a execução as estatísticas escolhidas e informações
necessárias para a validação da execução destes protocolos também pode ser parametrizada e estendida.

Neste documento propomos uma extensão desta ferramenta para melhor simular e fornecer diferentes conjuntos de ambientes de execução
permitindo a parametrização da camada de rede do simulador. Isto vai permitir a definição de protocolos onde esta camada vai ser construida, 
tanto com uma rede estruturada ou não estruturada, ou permitindo à rede a execução de algoritmos de otimização para a mesma, permitindo assim o estudo
do desempenho, a correção destes protocolos num ambiente dinâmico, em diferentes camadas de rede que vão independentemente responder a comportamentos
bizantinos ou mudanças na camada de rede.

% Palavras-chave do resumo em Português
% \begin{keywords}
% Palavra-chave 1, Palavra-chave 2, Palavra-chave 3, Palavra-chave 4
% \end{keywords}
\keywords{
  Tecnologia de Registo Distribuído
}
% to add an extra black line
